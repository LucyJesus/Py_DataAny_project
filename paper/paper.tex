\documentclass{ctexart}

\usepackage[dvipsnames]{xcolor}
\usepackage{hyperref}
\usepackage{amsmath}
\usepackage{amsthm}
\usepackage{graphicx}
\usepackage{graphicx}

\hypersetup{colorlinks=true, pdfstartview=FitV, 
linkcolor=BrickRed,citecolor=BrickRed, urlcolor=BrickRed}

\ctexset{
section = {
format = \raggedright\Large\bfseries,
}
}

\title{没想好标题}
\author{杨施诚 \and 罗鹍 \and 杨涛}
\date{\today}

\begin{document}
    \maketitle
    \begin{abstract}
        {\noindent 这里是摘要这里是摘要
        这里是摘要这里是摘要这里是摘要这里是摘要
        这里是摘要这里是摘要这里是摘要这里是摘要
        这里是摘要这里是摘要这里是摘要这里是摘要
        这里是摘要这里是摘要这里是摘要这里是摘要
        这里是摘要这里是摘要这里是摘要这里是摘要
        这里是摘要}
        \par\noindent \textbf{关键词:}关注taffy谢谢喵!
    \end{abstract}

    \newpage
    \tableofcontents

    \newpage
    \section{第一部分}
    哈哈哈,孙笑川真帅!
    
    \section{第二部分}
    
    \section{第三部分}
    
    \section{可视化操作}
    \noindent 主要使用pandas库对文件\textbf{birth.csv,birth\_rt.csv,death.csv,death\_rt.csv}文件进行可视化操作

    \newpage
    \appendix
    \section{附录一}
    这里是附录,这里没有 taffy。

    %\newpage
    %\bibliographystyle{alpha}
	%\bibliography{refs}
    %参考文献,写在refs.bib

    
\end{document}
